\section{Conclusiones y Recomendaciones}

Este capítulo presenta las conclusiones derivadas del desarrollo e implementación de Irakani Builder, evaluando el cumplimiento de los objetivos planteados y el impacto del proyecto. Asimismo, se ofrecen recomendaciones para el trabajo futuro y mejoras potenciales de la plataforma.

\subsection{Conclusiones}

\subsubsection{Cumplimiento del objetivo general}

El objetivo general del proyecto era desarrollar una plataforma web moderna que permitiera la creación rápida de aplicaciones mediante un editor visual, integrando capacidades de inteligencia artificial para asistir en el proceso de desarrollo y reducir significativamente los tiempos de implementación.

\textbf{Logros alcanzados:}

El proyecto ha cumplido satisfactoriamente con el objetivo general planteado, como lo demuestran los siguientes resultados:

\begin{enumerate}
    \item \textbf{Reducción de tiempos de desarrollo}: El 100\% de los usuarios encuestados reporta ahorros de tiempo superiores al 30\%, con un tercio de ellos experimentando ahorros superiores al 50\%. Si bien el objetivo inicial planteaba una reducción del 45-50\%, los resultados muestran que la mayoría de usuarios (66.7\%) alcanzó ahorros en el rango de 30-50\%, lo cual representa un avance significativo aunque con margen de mejora para alcanzar consistentemente el objetivo proyectado.
    
    \item \textbf{Integración exitosa de IA}: El asistente de inteligencia artificial obtuvo una calificación promedio de 4.33/5 en utilidad, demostrando que la integración de capacidades de IA es efectiva y altamente valorada por los usuarios. La baja frecuencia de alucinaciones (100\% reporta rara vez) confirma la fiabilidad del sistema.
    
    \item \textbf{Plataforma web funcional}: Se logró desarrollar una plataforma web moderna completamente renovada, accesible desde cualquier dispositivo con navegador, superando las limitaciones tecnológicas de la versión anterior (app.irakani.com) que utilizaba frameworks obsoletos.
    
    \item \textbf{Editor visual operativo}: Se implementó un editor visual completo con interfaz de arrastrar y soltar, paneles redimensionables, preview en tiempo real y gestión visual de componentes, cumpliendo con los requisitos funcionales establecidos.
    
    \item \textbf{Automatización de tareas repetitivas}: La generación automática de iconos y formularios fue bien recibida, con el 66.7\% de usuarios confirmando que acelera definitivamente su flujo de trabajo.
    
    \item \textbf{Integración de herramientas profesionales}: La incorporación de Monaco Editor proporciona una experiencia de edición de código de nivel profesional directamente en el navegador, con una calificación promedio de 3.67/5.
\end{enumerate}

\textbf{Validación mediante métricas:}

Los resultados cuantitativos obtenidos de la encuesta de satisfacción validan el cumplimiento del objetivo:

\begin{itemize}
    \item Promedio general de satisfacción: 3.67/5
    \item Ahorro de tiempo: 100\% de usuarios reporta >30\%
    \item Utilidad del asistente de IA: 4.33/5
    \item Fiabilidad de la IA: 0\% de alucinaciones frecuentes
    \item Adopción de automatización: 100\% encuentra útil la generación automática
\end{itemize}

Estos indicadores demuestran que Irakani Builder no solo cumple con los objetivos técnicos planteados, sino que también genera valor real y medible para sus usuarios.

\textbf{Impacto en el proceso de desarrollo:}

La plataforma ha transformado el proceso de desarrollo de aplicaciones en Irakani de las siguientes maneras:

\begin{itemize}
    \item \textbf{Democratización del desarrollo}: La interfaz visual y el asistente de IA reducen la barrera de entrada, permitiendo que usuarios con diferentes niveles de experiencia puedan crear aplicaciones.
    
    \item \textbf{Aceleración del prototipado}: La combinación de componentes predefinidos, generación automática y preview en tiempo real permite crear prototipos funcionales en una fracción del tiempo anterior.
    
    \item \textbf{Reducción de errores}: La generación de código mediante IA y las validaciones integradas reducen errores comunes de sintaxis y estructura.
    
    \item \textbf{Mejora en la consistencia}: El uso de componentes estandarizados y generación automática asegura mayor consistencia en las aplicaciones desarrolladas.
\end{itemize}

\subsubsection{Impacto en la deuda técnica}

La migración de la plataforma anterior a Irakani Builder ha tenido un impacto significativo en la gestión de la deuda técnica de la organización.

\textbf{Reducción de deuda técnica:}

\begin{enumerate}
    \item \textbf{Modernización de la arquitectura}: La transición de una aplicación de escritorio a una arquitectura web moderna basada en microservicios reduce significativamente la deuda técnica acumulada. La nueva arquitectura es más mantenible, escalable y alineada con las mejores prácticas actuales.
    
    \item \textbf{Actualización de tecnologías}: El uso de tecnologías modernas (React, Node.js, Express, MySQL, SQL Server) en lugar de frameworks obsoletos facilita el mantenimiento futuro y la incorporación de nuevos desarrolladores al equipo.
    
    \item \textbf{Mejora en la mantenibilidad}: La arquitectura de microservicios permite actualizar y mantener componentes individuales sin afectar el sistema completo, reduciendo el riesgo de regresiones.
    
    \item \textbf{Documentación implícita}: El código generado por IA tiende a seguir patrones estándar y buenas prácticas, lo que facilita su comprensión y mantenimiento.
\end{enumerate}

\textbf{Prevención de nueva deuda técnica:}

El proyecto incorpora mecanismos para prevenir la acumulación de nueva deuda técnica:

\begin{itemize}
    \item \textbf{Generación de código estandarizado}: La IA genera código siguiendo patrones consistentes, reduciendo la variabilidad y mejorando la mantenibilidad.
    
    \item \textbf{Validaciones automáticas}: El sistema incluye validaciones que previenen la introducción de código problemático.
    
    \item \textbf{Arquitectura modular}: La separación en microservicios facilita la evolución independiente de cada componente.
    
    \item \textbf{Tecnologías con soporte activo}: El uso de tecnologías ampliamente adoptadas y con comunidades activas asegura soporte a largo plazo.
\end{itemize}

\textbf{Desafíos pendientes:}

A pesar de los avances, existen áreas que requieren atención continua:

\begin{itemize}
    \item \textbf{Calidad del código generado}: Con una calificación de 3.33/5, el código generado por IA requiere mejoras para reducir la necesidad de correcciones manuales.
    
    \item \textbf{Documentación y curva de aprendizaje}: La calificación moderada en intuitividad (3.33/5) sugiere la necesidad de mejor documentación y materiales de capacitación. El objetivo específico de optimizar la curva de aprendizaje en un 40\% no pudo ser medido en esta fase inicial, por lo que se requiere establecer métricas base y realizar evaluaciones comparativas en futuras iteraciones.
    
    \item \textbf{Pruebas automatizadas}: Es necesario implementar una suite completa de pruebas automatizadas para asegurar la calidad a largo plazo.
    
    \item \textbf{Monitoreo y observabilidad}: Se requiere implementar sistemas de monitoreo más robustos para detectar y resolver problemas proactivamente.
\end{itemize}

\subsection{Recomendaciones}

Con base en los resultados obtenidos y las áreas de mejora identificadas, se presentan las siguientes recomendaciones para el desarrollo futuro de la plataforma.

\subsubsection{Trabajo a futuro y mejoras potenciales}

\textbf{Mejoras prioritarias en IA y usabilidad:}

\begin{itemize}
    \item \textbf{Refinamiento del asistente de IA}: Optimizar prompts y validaciones automáticas para mejorar la calidad del código generado (objetivo: >4/5), implementando aprendizaje contextual basado en correcciones de usuarios.
    
    \item \textbf{Mejora de la experiencia de usuario}: Desarrollar onboarding interactivo, documentación contextual y plantillas predefinidas para alcanzar una calificación superior a 4/5 en intuitividad.
\end{itemize}

\textbf{Funcionalidades sugeridas por usuarios:}

\begin{itemize}
    \item \textbf{Duplicación de espacios con IA}: Clonar espacios de trabajo completos con adaptación automática del contenido.
    
    \item \textbf{Sistema avanzado de indicadores}: Herramientas sofisticadas para crear, modificar y visualizar KPIs con capacidades predictivas.
\end{itemize}

\textbf{Mejoras técnicas y operativas:}

\begin{itemize}
    \item \textbf{Optimización de rendimiento}: Mejorar tiempos de carga y respuesta, especialmente en proyectos grandes.
    
    \item \textbf{Colaboración y versionamiento}: Implementar edición colaborativa en tiempo real y control de versiones integrado.
    
    \item \textbf{Escalabilidad}: Optimizar consumo de AWS Bedrock mediante caché y reutilización de respuestas, implementar escalado automático.
    
    \item \textbf{Seguridad}: Realizar auditorías periódicas, mejorar gestión de permisos y asegurar cifrado de datos.
\end{itemize}

\textbf{Priorización de mejoras:}

\begin{itemize}
    \item \textbf{Corto plazo (1-3 meses):} Calidad del código generado, documentación, onboarding y plantillas predefinidas.
    \item \textbf{Mediano plazo (3-6 meses):} Duplicación de espacios con IA, sistema de indicadores, control de versiones y colaboración en tiempo real.
    \item \textbf{Largo plazo (6-12 meses):} Modo offline (PWA), optimización de costos con modelos locales e integración CI/CD.
\end{itemize}

\textbf{Consideraciones finales:}

Irakani Builder representa un avance significativo en el desarrollo de aplicaciones dentro de la organización. Los resultados validan la viabilidad de la plataforma y su capacidad para generar valor real. El éxito del proyecto demuestra que la integración de IA en herramientas de desarrollo es técnicamente viable, altamente valorada por los usuarios y genera mejoras medibles en productividad. La clave del éxito futuro será mantener el enfoque en las necesidades de los usuarios, iterar continuamente basándose en feedback real, y equilibrar la innovación con la estabilidad del sistema.
